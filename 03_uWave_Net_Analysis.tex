% !TeX root = Microwave.tex
\chapter{Microwave Elements and Networks 微波元件与网络分析}
\section{S参数的引入}
\begin{equation*}
    \begin{array}{cc}
        \makecell[cc]{\mbox{A参数}\begin{cases}
            \mbox{只适用于双端口(分组)}\\
            \mbox{适用于传输问题讨论}\\
            \mbox{端口量是电压和电流}\\
        \end{cases}}&
        \makecell[cc]{\mbox{S参数}\begin{cases}
            \mbox{多端口}\\
            \mbox{任意输入输出}\\
            \mbox{波参数直观反映入射和反射}
        \end{cases}}
    \end{array}
\end{equation*}

    \subsection{入射波和反射波}
    由传输线理论的本征模思想,电压电流的波动方程可以视作入射波和反射波的叠加,即\hyperref[Equ: Eigenmode u(z) and i(z)]{式(\ref*{Equ: Eigenmode u(z) and i(z)})}的形式。反过来,则可以用叠加后的电压电流表示入射波和反射波:
    \begin{subequations}
        \begin{numcases}{}
            \mbox{入射波:}u^+\mathrm{e}^{-\gamma z}=\frac{1}{2}(u+iZ_0) \\
            \mbox{反射波:}u^-\mathrm{e}^{\gamma z}=\frac{1}{2}(u-iZ_0)
        \end{numcases}
    \end{subequations}

    作量纲归一化处理:
    \begin{subequations}
        \begin{numcases}{}
            \mbox{入射波$a$:}a=\frac{u^+\mathrm{e}^{-\gamma z}}{\sqrt{Z_0}}
                =\frac{1}{2}\left(\frac{u}{\sqrt{Z_0}}+i\sqrt{Z_0}\right)
                =\frac{1}{2}(\bar{u}+\bar{i}) \\
            \mbox{散射波$b$:}b=\frac{u^-\mathrm{e}^{\gamma z}}{\sqrt{Z_0}}
                \;=\frac{1}{2}\left(\frac{u}{\sqrt{Z_0}}-i\sqrt{Z_0}\right)
                =\frac{1}{2}(\bar{u}-\bar{i})
        \end{numcases}
    \end{subequations}
    归一化后,用$a,b$表示功率时式中不含有$Z_0$。


    称其中
    \begin{subequations}
        \begin{numcases}{}
            \mbox{归一化电压:} \bar{u}=\frac{u}{\sqrt{Z_0}}=a+b\\
            \mbox{归一化电流:} \bar{i}=i\sqrt{Z_0}=a-b
        \end{numcases}
    \end{subequations}

    传输功率
    \begin{equation}
        \begin{aligned}
            p&=\frac{1}{2}\Re[\bar{u}\bar{i}^*]\\
            &=\frac{1}{2}\Re[(a+b)(a^*-b^*)]\\
            &=\frac{1}{2}(aa^*-bb^*)+\frac{1}{2}\Re[a^*b-ab^*]\\
            &=\frac{1}{2}(aa^*-bb^*)
        \end{aligned}
    \end{equation}
    其中$\frac{1}{2}aa^*$和$\frac{1}{2}bb^*$恰好对应入射功率和散射功率。

    \subsection{S参数的定义和物理意义}
    设有$n$端口网络,每个端口均有输入量和输出量,分别用前面所定义的入射波$a$和散射波$b$表示。则$\bm{S}$散射参数$S_{i1},\cdots,S_{in}$定义了每一个散射波$b_i$如何由各个入射波$a_1,\cdots,a_n$加权线性组合得到。
    \begin{equation}
        \begin{bmatrix}
            b_1\\b_2\\\vdots\\b_n
        \end{bmatrix}
        =\begin{bmatrix}
            S_{11}&S_{12}&\cdots&S_{1n}\\
            S_{21}&S_{22}&\cdots&S_{2n}\\
            \vdots&\vdots&\vdots&\vdots\\
            S_{n1}&S_{n2}&\cdots&S_{nn}
        \end{bmatrix}
        \begin{bmatrix}
            a_1\\a_2\\\vdots\\a_n
        \end{bmatrix}
    \end{equation}


    \subsection{网络性质的S参数描述}
        \begin{enumerate}
            \item 对称性质
            \begin{equation}
                S_{ii}=S_{jj}\quad\mbox{($i$端口与$j$端口对称)}
            \end{equation}
            \item 无耗性质\footnote{$[\,\cdot\,]^\dagger$为Hermite符号,表示矩阵的共轭转置或转置共轭。}
            \begin{equation}
                \bm{S}^\dagger\bm{S}=\bm{I}
            \end{equation}
            此性质又称为$\bm{S}$参数矩阵的幺正性。


            特殊地,当网络为二端口网络时,代入可得:
            \begin{equation}
                \begin{bmatrix}
                    S_{11}^*&S_{21}^*\\
                    S_{12}^*&S_{22}^*
                \end{bmatrix}
                \begin{bmatrix}
                    S_{11}&S_{12}\\
                    S_{21}&S_{22}
                \end{bmatrix}
                =\begin{bmatrix}
                    1&0\\
                    0&1
                \end{bmatrix}
            \end{equation}
            上式等价于
            \begin{equation}
                \left\{\begin{aligned}
                    &\begin{subequations}
                        \mbox{振幅条件}
                        \left\{\begin{aligned}
                            |S_{11}|^2+|S_{12}|^2=1 \\
                            |S_{21}|^2+|S_{22}|^2=1
                        \end{aligned}\right.
                    \end{subequations}\\
                    &\begin{subequations}
                        \mbox{相位条件}
                        \left\{\begin{aligned}
                            S_{11}^*S_{12}+S_{21}^*S_{22}\\
                            S_{11}S_{12}^*+S_{21}S_{22}^*
                        \end{aligned}\right.
                    \end{subequations}
                \end{aligned}\right.
            \end{equation}
            这是一个欠定方程组,可以推知:
            \begin{subequations}
                \begin{numcases}{\mbox{}}
                    |S_{11}|=|S_{22}| \\
                    |S_{12}|=|S_{21}| \\
                    (\varphi_{12}+\varphi_{21})-(\varphi_{11}+\varphi_{22})=\pm\pi
                \end{numcases}
            \end{subequations}
            \item 互易性质
            \begin{equation}
                S_{ij}=S_{ji}\quad\mbox{($i$端口与$j$端口互易)}
            \end{equation}
            \item 反射系数转换性质
            \begin{equation}
                \varGamma_\mathrm{in}=S_{11}+\frac{S_{12}S_{21}\varGamma_L}{1-S_{22}\varGamma_L}
            \end{equation}
            其中,$\varGamma_L=\frac{a_2}{b_2}$为负载反射系数
        \end{enumerate}


\section{S参数的变换定理}
    \subsection{负载变换}
    \subsection{A参数矩阵转化为S参数矩阵}
    % \paragraph{}
    
    归一化后的$\bm{A}$参数矩阵转$\bm{S}$参数矩阵:
    \begin{equation}
        \bm{S}=\frac{1}{A_{11}+A_{12}+A_{21}+A_{22}}
        \begin{bmatrix}
            A_{11}+A_{12}-A_{21}-A_{22}&2|\bm{A}|\\
            2&A_{12}+A_{22}-A_{21}-A_{11}
        \end{bmatrix}
    \end{equation}

\section{单端口元件}
    \subsection{匹配元件}
    \subsection{短路元件}
    \subsection{失配元件}

