% !TeX root = Microwave.tex
\chapter{Microwave Elements and Networks 微波元件与网络分析}
\setlength{\parindent}{2\ccwd}
\section{S参数的引入}\label{Sec: S参数的引入}
\begin{equation*}
    \begin{array}{cc}
        \makecell[cc]{\mbox{A参数}\begin{cases}
            \mbox{只适用于双端口(分组)}\\
            \mbox{适用于传输问题讨论}\\
            \mbox{端口量是电压和电流}\\
        \end{cases}}&
        \makecell[cc]{\mbox{S参数}\begin{cases}
            \mbox{多端口}\\
            \mbox{任意输入输出}\\
            \mbox{波参数直观反映入射和反射}
        \end{cases}}
    \end{array}
\end{equation*}

    \subsection{入射波和反射波}
    由传输线理论的本征模思想,电压电流的波动方程可以视作入射波和反射波的叠加,即\hyperref[Equ: Eigenmode u(z) and i(z)]{式(\ref*{Equ: Eigenmode u(z) and i(z)})}的形式。反过来,则可以用叠加后的电压电流表示入射波和反射波:
    \begin{subequations}
        \begin{numcases}{}
            \mbox{入射波:}u^+\mathrm{e}^{-\gamma z}=\frac{1}{2}(u+iZ_0) \\
            \mbox{反射波:}u^-\mathrm{e}^{\gamma z}=\frac{1}{2}(u-iZ_0)
        \end{numcases}
    \end{subequations}

    作量纲归一化处理:
    \begin{subequations}
        \begin{numcases}{}
            \mbox{入射波$a$:}a=\frac{u^+\mathrm{e}^{-\gamma z}}{\sqrt{Z_0}}
                =\frac{1}{2}\left(\frac{u}{\sqrt{Z_0}}+i\sqrt{Z_0}\right)
                =\frac{1}{2}(\bar{u}+\bar{i}) \\
            \mbox{散射波$b$:}b=\frac{u^-\mathrm{e}^{\gamma z}}{\sqrt{Z_0}}
                \;=\frac{1}{2}\left(\frac{u}{\sqrt{Z_0}}-i\sqrt{Z_0}\right)
                =\frac{1}{2}(\bar{u}-\bar{i})
        \end{numcases}
    \end{subequations}
    归一化后,用$a,b$表示功率时式中不含有$Z_0$。


    称其中
    \begin{subequations}
        \begin{numcases}{}
            \mbox{归一化电压:} \bar{u}=\frac{u}{\sqrt{Z_0}}=a+b\\
            \mbox{归一化电流:} \bar{i}=i\sqrt{Z_0}=a-b
        \end{numcases}
    \end{subequations}

    传输功率
    \begin{equation}
        \begin{aligned}
            p&=\frac{1}{2}\Re[\bar{u}\bar{i}^*]\\
            &=\frac{1}{2}\Re[(a+b)(a^*-b^*)]\\
            &=\frac{1}{2}(aa^*-bb^*)+\frac{1}{2}\Re[a^*b-ab^*]\\
            &=\frac{1}{2}(aa^*-bb^*)
        \end{aligned}
    \end{equation}
    其中$\frac{1}{2}aa^*$和$\frac{1}{2}bb^*$恰好对应入射功率和散射功率。

    \subsection{S参数的定义和物理意义}
    设有$n$端口网络,每个端口均有输入量和输出量,分别用前面所定义的入射波$a$和散射波$b$表示。则$\bm{S}$散射参数$S_{i1},\cdots,S_{in}$定义了每一个散射波$b_i$如何由各个入射波$a_1,\cdots,a_n$加权线性组合得到。
    \begin{equation}
        \begin{bmatrix}
            b_1\\b_2\\\vdots\\b_n
        \end{bmatrix}
        =\begin{bmatrix}
            S_{11}&S_{12}&\cdots&S_{1n}\\
            S_{21}&S_{22}&\cdots&S_{2n}\\
            \vdots&\vdots&\vdots&\vdots\\
            S_{n1}&S_{n2}&\cdots&S_{nn}
        \end{bmatrix}
        \begin{bmatrix}
            a_1\\a_2\\\vdots\\a_n
        \end{bmatrix}
    \end{equation}


    \subsection{网络性质的S参数描述}
        \begin{enumerate}
            \item 对称性质
            \begin{equation}
                S_{ii}=S_{jj}\quad\mbox{($i$端口与$j$端口对称)}
            \end{equation}
            \item 无耗性质\footnote{$[\,\cdot\,]^\dagger$为Hermite符号,表示矩阵的共轭转置或转置共轭。}
            \begin{equation}
                \bm{S}^\dagger\bm{S}=\bm{I}
            \end{equation}
            此性质又称为$\bm{S}$参数矩阵的幺正性。


            特殊地,当网络为二端口网络时,代入可得:
            \begin{equation}
                \begin{bmatrix}
                    S_{11}^*&S_{21}^*\\
                    S_{12}^*&S_{22}^*
                \end{bmatrix}
                \begin{bmatrix}
                    S_{11}&S_{12}\\
                    S_{21}&S_{22}
                \end{bmatrix}
                =\begin{bmatrix}
                    1&0\\
                    0&1
                \end{bmatrix}
            \end{equation}
            上式等价于
            \begin{equation}
                \left\{\begin{aligned}
                    &\begin{subequations}
                        \mbox{振幅条件}
                        \left\{\begin{aligned}
                            |S_{11}|^2+|S_{12}|^2=1 \\
                            |S_{21}|^2+|S_{22}|^2=1
                        \end{aligned}\right.
                    \end{subequations}\\
                    &\begin{subequations}
                        \mbox{相位条件}
                        \left\{\begin{aligned}
                            S_{11}^*S_{12}+S_{21}^*S_{22}\\
                            S_{11}S_{12}^*+S_{21}S_{22}^*
                        \end{aligned}\right.
                    \end{subequations}
                \end{aligned}\right.
            \end{equation}
            这是一个欠定方程组,可以推知:
            \begin{subequations}
                \begin{numcases}{\mbox{}}
                    |S_{11}|=|S_{22}| \\
                    |S_{12}|=|S_{21}| \\
                    (\varphi_{12}+\varphi_{21})-(\varphi_{11}+\varphi_{22})=\pm\pi
                \end{numcases}
            \end{subequations}
            \item 互易性质
            \begin{equation}
                S_{ij}=S_{ji}\quad\mbox{($i$端口与$j$端口互易)}
            \end{equation}

        \end{enumerate}


\section{S参数的变换定理}

    \subsection{负载变换}

    \subsection{反射系数变换}
    反射系数在经过一个由S参数描述的网络之后,产生如下变换:
    \begin{equation}
        \varGamma_\mathrm{in}=S_{11}+\frac{S_{12}S_{21}\varGamma_L}{1-S_{22}\varGamma_L}
    \end{equation}
    其中,$\varGamma_L=\frac{a_2}{b_2}$为负载反射系数    
    \subsection{A参数矩阵转化为S参数矩阵}
    % \paragraph{}
    需要先对A参数矩阵做阻抗归一化。最一般的情况如\hyperref[Fig: A参数矩阵的阻抗归一化]{图\ref*{Fig: A参数矩阵的阻抗归一化}}所示,网络的两个端口具有不同的特性阻抗。

    \begin{figure}[h!]
        \begin{center}
            \begin{tikzpicture}[
                    circuit ee IEC,
                    % x=3cm,y=2cm,
                    semithick,
                    every info/.style={font=\footnotesize},
                    small circuit symbols,
                    % set resistor graphic=var resistor IEC graphic,
                    % set diode graphic=var diode IEC graphic,
                    set make contact graphic = var make contact IEC graphic,
                    circuit declare symbol= module, set module graphic=
                        {draw, shape=rectangle, minimum width=3cm, minimum height=3cm, text width = 2cm, align = flush center, fill=white},
                    ]
                %%
                \node [circle, draw=black, inner sep=1pt, minimum size=3pt, label=left:{\color{blue}$-$}] (P1far_N) at (-3.5,-1) {};
                \node [circle, draw=black, inner sep=1pt, minimum size=3pt, label=left:{\color{blue}$+$}] (P1far_P) at (-3.5,1) {};
                \node [circle, draw=black, inner sep=1pt, minimum size=3pt, label=right:{\color{blue}$-$}] (P2far_N) at (3.5,-1) {};
                \node [circle, draw=black, inner sep=1pt, minimum size=3pt, label=right:{\color{blue}$+$}] (P2far_P) at (3.5,1) {};
                
                \node [contact] (P1near_N) at (-1.5,-1) {};
                \node [contact] (P1near_P) at (-1.5,1) {};
                \node [contact] (P2near_N) at (1.5,-1) {};
                \node [contact] (P2near_P) at (1.5,1) {};

                \draw (P1far_P) to (P1near_P);
                \draw (P1far_N) to (P1near_N);
                
                \draw (P2far_P) to (P2near_P);
                \draw (P2far_N) to (P2near_N);

                % \node [module] (网络A) at (0,0) {$\displaystyle \bm{A}=\qquad\qquad \begin{bmatrix}A_{11}&A_{12}\\ A_{21}&A_{22}\end{bmatrix}$};
                \node [module] (网络A) at (0,0) {$[\bm{A}]$};

                \draw [draw=none, fill=lightgray, semitransparent] (P1near_N)++(0,-0.5) rectangle ++(-2.6,3);
                \draw [draw=none, fill=gray, semitransparent] (P2near_N)++(0,-0.5) rectangle ++(2.6,3);

                \draw[->, blue] (P1far_P) ++(0.8cm,0) -- node[above]{$i_1$} ++(0.8cm,0);
                \draw[->, blue] (P2near_P) ++(0.8cm,0) -- node[above]{$i_2$} ++(0.8cm,0);
                
                \draw (-3.8,0) node[label=right:{$\quad Z_{01}$}] {\color{blue} $u_1$};
                \draw (3.8,0) node[label=left:{$Z_{02}\quad $}] {\color{blue} $u_2$};

                % \draw (8,0) node[] {$\displaystyle \begin{bmatrix}
                %     u_1\\i_1
                % \end{bmatrix}=\bm{A}\begin{bmatrix}
                %     u_2\\i_2
                % \end{bmatrix}$};
            \end{tikzpicture}
        \end{center}
        \caption{\kaishu A参数矩阵的阻抗归一化}\label{Fig: A参数矩阵的阻抗归一化}
    \end{figure}

    \begin{theorem}{A参数矩阵的阻抗归一化}{A参数矩阵的阻抗归一化}
        一般地,记一个二端口网络的A参数矩阵为:
        \begin{equation*}
            \bm{A}=\begin{bmatrix}
                A_{11}&A_{12}\\
                A_{21}&A_{22}
            \end{bmatrix}
        \end{equation*}
        记其阻抗归一化A参数矩阵为$\bm{a}$,
        \begin{equation*}
            \bm{a}=\begin{bmatrix}
                a_{11}&a_{12}\\
                a_{21}&a_{22}
            \end{bmatrix}
        \end{equation*}
        则有:
        \begin{equation}
            \begin{bmatrix}
                a{11}&a_{12}\\
                a_{21}&a_{22}
            \end{bmatrix}    
            =\begin{bmatrix}
                \dfrac{\sqrt{Z_{02}}A_{11}}{\sqrt{Z_{01}}} & \dfrac{A_{12}}{\sqrt{Z_{01}Z_{02}}}\\
                \sqrt{Z_{01}Z_{02}}A_{21} & \dfrac{\sqrt{Z_{01}}A_{22}}{\sqrt{Z_{02}}}
            \end{bmatrix}
        \end{equation}
    \end{theorem}

    \begin{tcbproof}\begin{proof}
        根据A参数矩阵的定义和阻抗归一化波的概念就可以推导出阻抗归一化A参数矩阵的表达式。将归一化电压和电流表示为:
        \begin{equation}
            \begin{aligned}
                \begin{bmatrix}
                    \bar{u}_1\\
                    \bar{i}_1
                \end{bmatrix}
                &=\begin{bmatrix}
                    \dfrac{u_1}{\sqrt{Z_{01}}}\\
                    i_1\sqrt{Z_{01}}
                \end{bmatrix}\\
                &=\begin{bmatrix}
                    \dfrac{A_{11}}{\sqrt{Z_{01}}}u_2+\dfrac{A_{12}}{Z_{01}}i_2\\
                    A_{21}\sqrt{Z_{01}}u_2+A_{22}\sqrt{Z_{01}}i_2
                \end{bmatrix}\\
                &=\begin{bmatrix}
                    \dfrac{\sqrt{Z_{02}}A_{11}}{\sqrt{Z_{01}}}\bar{u}_2+\dfrac{A_{12}}{\sqrt{Z_{01}Z_{02}}}\bar{i}_2\\
                    A_{21}\sqrt{Z_{01}Z_{02}}\bar{u}_2+\dfrac{\sqrt{Z_{01}}A_{22}}{\sqrt{Z_{02}}}\bar{i}_2
                \end{bmatrix}\\
                &=\begin{bmatrix}
                    \dfrac{\sqrt{Z_{02}}A_{11}}{\sqrt{Z_{01}}} & \dfrac{A_{12}}{\sqrt{Z_{01}Z_{02}}}\\
                    A_{21}\sqrt{Z_{01}Z_{02}} & \dfrac{\sqrt{Z_{01}}A_{22}}{\sqrt{Z_{02}}}
                \end{bmatrix}
                \begin{bmatrix}
                    \bar{u}_2\\
                    \bar{i}_2
                \end{bmatrix}
                =\begin{bmatrix}
                    a_{11}&a_{12}\\
                    a_{21}&a_{22}
                \end{bmatrix}
                \begin{bmatrix}
                    \bar{u}_2\\
                    \bar{i}_2                
                \end{bmatrix}
                =\bm{a}
                \begin{bmatrix}
                    \bar{u}_2\\
                    \bar{i}_2                
                \end{bmatrix}
            \end{aligned}
        \end{equation}
        其中$\bm{a}$表示阻抗归一化的A参数矩阵。        
    \end{proof}\end{tcbproof}


    归一化后的A参数矩阵转S参数矩阵:
    \begin{equation}
        \bm{S}=\frac{1}{a_{11}+a_{12}+a_{21}+a_{22}}
        \begin{bmatrix}
            a_{11}+a_{12}-a_{21}-a_{22}&2|\bm{a}|\\
            2&a_{12}+a_{22}-a_{21}-a_{11}
        \end{bmatrix}
    \end{equation}

\section{单端口元件}
    \subsection{匹配元件}
    \subsection{短路元件}
    \subsection{失配元件}

