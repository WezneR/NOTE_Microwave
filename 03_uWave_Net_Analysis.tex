% !TeX root = Microwave.tex
\chapter{Microwave Elements and Networks 微波元件与网络分析}
\setlength{\parindent}{2\ccwd}
\section{S参数的引入}\label{Sec: S参数的引入}
\begin{equation*}
    \begin{array}{cc}
        \makecell[cc]{\mbox{A参数}\begin{cases}
            \mbox{只适用于双端口(分组)}\\
            \mbox{适用于传输问题讨论}\\
            \mbox{端口量是电压和电流}\\
        \end{cases}}&
        \makecell[cc]{\mbox{S参数}\begin{cases}
            \mbox{多端口}\\
            \mbox{任意输入输出}\\
            \mbox{波参数直观反映入射和反射}
        \end{cases}}
    \end{array}
\end{equation*}

    \subsection{入射波和反射波}
    由传输线理论的本征模思想,电压电流的波动方程可以视作入射波和反射波的叠加,即\hyperref[Equ: Eigenmode u(z) and i(z)]{式(\ref*{Equ: Eigenmode u(z) and i(z)})}的形式。反过来,则可以用叠加后的电压电流表示入射波和反射波:
    \begin{subequations}
        \begin{numcases}{}
            \mbox{入射波:}u^+\mathrm{e}^{-\gamma z}=\frac{1}{2}(u+iZ_0) \\
            \mbox{反射波:}u^-\mathrm{e}^{\gamma z}=\frac{1}{2}(u-iZ_0)
        \end{numcases}
    \end{subequations}

    作量纲归一化处理:
    \begin{subequations}
        \begin{numcases}{}
            \mbox{入射波$a$:}a=\frac{u^+\mathrm{e}^{-\gamma z}}{\sqrt{Z_0}}
                =\frac{1}{2}\left(\frac{u}{\sqrt{Z_0}}+i\sqrt{Z_0}\right)
                =\frac{1}{2}(\bar{u}+\bar{i}) \\
            \mbox{散射波$b$:}b=\frac{u^-\mathrm{e}^{\gamma z}}{\sqrt{Z_0}}
                \;=\frac{1}{2}\left(\frac{u}{\sqrt{Z_0}}-i\sqrt{Z_0}\right)
                =\frac{1}{2}(\bar{u}-\bar{i})
        \end{numcases}
    \end{subequations}
    归一化后,用$a,b$表示功率时式中不含有$Z_0$。


    称其中
    \begin{subequations}
        \begin{numcases}{}
            \mbox{归一化电压:} \bar{u}=\frac{u}{\sqrt{Z_0}}=a+b\\
            \mbox{归一化电流:} \bar{i}=i\sqrt{Z_0}=a-b
        \end{numcases}
    \end{subequations}

    传输功率
    \begin{equation}
        \begin{aligned}
            p&=\frac{1}{2}\Re[\bar{u}\bar{i}^*]\\
            &=\frac{1}{2}\Re[(a+b)(a^*-b^*)]\\
            &=\frac{1}{2}(aa^*-bb^*)+\frac{1}{2}\Re[a^*b-ab^*]\\
            &=\frac{1}{2}(aa^*-bb^*)
        \end{aligned}
    \end{equation}
    其中$\frac{1}{2}aa^*$和$\frac{1}{2}bb^*$恰好对应入射功率和散射功率。

    \subsection{S参数的定义和物理意义}
    设有$n$端口网络,每个端口均有输入量和输出量,分别用前面所定义的入射波$a$和散射波$b$表示。则$\bm{S}$散射参数$S_{i1},\cdots,S_{in}$定义了每一个散射波$b_i$如何由各个入射波$a_1,\cdots,a_n$加权线性组合得到。
    \begin{equation}
        \begin{bmatrix}
            b_1\\b_2\\\vdots\\b_n
        \end{bmatrix}
        =\begin{bmatrix}
            S_{11}&S_{12}&\cdots&S_{1n}\\
            S_{21}&S_{22}&\cdots&S_{2n}\\
            \vdots&\vdots&\vdots&\vdots\\
            S_{n1}&S_{n2}&\cdots&S_{nn}
        \end{bmatrix}
        \begin{bmatrix}
            a_1\\a_2\\\vdots\\a_n
        \end{bmatrix}
    \end{equation}


    \subsection{网络性质的S参数描述}
        \begin{enumerate}
            \item 对称性质
            \begin{equation}
                S_{ii}=S_{jj}\quad\mbox{($i$端口与$j$端口对称)}
            \end{equation}
            \item 无耗性质\footnote{$[\,\cdot\,]^\dagger$为Hermite符号,表示矩阵的共轭转置或转置共轭。}
            \begin{equation}
                \bm{S}^\dagger\bm{S}=\bm{I}
            \end{equation}
            此性质又称为$\bm{S}$参数矩阵的幺正性。


            特殊地,当网络为二端口网络时,代入可得:
            \begin{equation}
                \begin{bmatrix}
                    S_{11}^*&S_{21}^*\\
                    S_{12}^*&S_{22}^*
                \end{bmatrix}
                \begin{bmatrix}
                    S_{11}&S_{12}\\
                    S_{21}&S_{22}
                \end{bmatrix}
                =\begin{bmatrix}
                    1&0\\
                    0&1
                \end{bmatrix}
            \end{equation}
            上式等价于
            \begin{equation}
                \left\{\begin{aligned}
                    &\begin{subequations}
                        \mbox{振幅条件}
                        \left\{\begin{aligned}
                            |S_{11}|^2+|S_{12}|^2=1 \\
                            |S_{21}|^2+|S_{22}|^2=1
                        \end{aligned}\right.
                    \end{subequations}\\
                    &\begin{subequations}
                        \mbox{相位条件}
                        \left\{\begin{aligned}
                            S_{11}^*S_{12}+S_{21}^*S_{22}\\
                            S_{11}S_{12}^*+S_{21}S_{22}^*
                        \end{aligned}\right.
                    \end{subequations}
                \end{aligned}\right.
            \end{equation}
            这是一个欠定方程组,可以推知:
            \begin{subequations}\label{Equ: 无耗二端口网络的S参数}
                \begin{numcases}{\mbox{}}
                    |S_{11}|=|S_{22}| \\
                    |S_{12}|=|S_{21}| \\
                    (\varphi_{12}+\varphi_{21})-(\varphi_{11}+\varphi_{22})=\pm\pi
                \end{numcases}
            \end{subequations}
            \item 互易性质
            \begin{equation}
                S_{ij}=S_{ji}\quad\mbox{($i$端口与$j$端口互易)}
            \end{equation}

        \end{enumerate}


    \subsection{A参数矩阵转化为S参数矩阵}
    需要先对A参数矩阵做阻抗归一化。最一般的情况如\hyperref[Fig: A参数矩阵的阻抗归一化]{图\ref*{Fig: A参数矩阵的阻抗归一化}}所示,网络的两个端口具有不同的特性阻抗。

    \begin{figure}[h!]
        \begin{center}
            \begin{tikzpicture}[
                    circuit ee IEC,
                    % x=3cm,y=2cm,
                    semithick,
                    every info/.style={font=\footnotesize},
                    small circuit symbols,
                    % set resistor graphic=var resistor IEC graphic,
                    % set diode graphic=var diode IEC graphic,
                    set make contact graphic = var make contact IEC graphic,
                    circuit declare symbol= module, set module graphic=
                        {draw, shape=rectangle, minimum width=3cm, minimum height=3cm, text width = 2cm, align = flush center, fill=white},
                    ]
                %%
                \node [circle, draw=black, inner sep=1pt, minimum size=3pt, label=left:{\color{blue}$-$}] (P1far_N) at (-3.5,-1) {};
                \node [circle, draw=black, inner sep=1pt, minimum size=3pt, label=left:{\color{blue}$+$}] (P1far_P) at (-3.5,1) {};
                \node [circle, draw=black, inner sep=1pt, minimum size=3pt, label=right:{\color{blue}$-$}] (P2far_N) at (3.5,-1) {};
                \node [circle, draw=black, inner sep=1pt, minimum size=3pt, label=right:{\color{blue}$+$}] (P2far_P) at (3.5,1) {};
                
                \node [contact] (P1near_N) at (-1.5,-1) {};
                \node [contact] (P1near_P) at (-1.5,1) {};
                \node [contact] (P2near_N) at (1.5,-1) {};
                \node [contact] (P2near_P) at (1.5,1) {};

                \draw (P1far_P) to (P1near_P);
                \draw (P1far_N) to (P1near_N);
                
                \draw (P2far_P) to (P2near_P);
                \draw (P2far_N) to (P2near_N);

                % \node [module] (网络A) at (0,0) {$\displaystyle \bm{A}=\qquad\qquad \begin{bmatrix}A_{11}&A_{12}\\ A_{21}&A_{22}\end{bmatrix}$};
                \node [module] (网络A) at (0,0) {$[\bm{A}]$};

                \draw [draw=none, fill=lightgray, semitransparent] (P1near_N)++(0,-0.5) rectangle ++(-2.6,3);
                \draw [draw=none, fill=gray, semitransparent] (P2near_N)++(0,-0.5) rectangle ++(2.6,3);

                \draw[->, blue] (P1far_P) ++(0.8cm,0) -- node[above]{$i_1$} ++(0.8cm,0);
                \draw[->, blue] (P2near_P) ++(0.8cm,0) -- node[above]{$i_2$} ++(0.8cm,0);
                
                \draw (-3.8,0) node[label=right:{$\quad Z_{01}$}] {\color{blue} $u_1$};
                \draw (3.8,0) node[label=left:{$Z_{02}\quad $}] {\color{blue} $u_2$};

            \end{tikzpicture}
        \end{center}
        \caption{\kaishu A参数矩阵的阻抗归一化}\label{Fig: A参数矩阵的阻抗归一化}
    \end{figure}

    \begin{theorem}
    {A参数矩阵的阻抗归一化}
    {A参数矩阵的阻抗归一化}
        一般地,记一个二端口网络的A参数矩阵为:
        \begin{equation*}
            \bm{A}=\begin{bmatrix}
                A_{11}&A_{12}\\
                A_{21}&A_{22}
            \end{bmatrix}
        \end{equation*}
        记其阻抗归一化A参数矩阵为$\bm{a}$,
        \begin{equation*}
            \bm{a}=\begin{bmatrix}
                a_{11}&a_{12}\\
                a_{21}&a_{22}
            \end{bmatrix}
        \end{equation*}
        则有:
        \begin{equation}
            \begin{bmatrix}
                a_{11}&a_{12}\\
                a_{21}&a_{22}
            \end{bmatrix}    
            =\begin{bmatrix}
                \dfrac{\sqrt{Z_{02}}A_{11}}{\sqrt{Z_{01}}} & \dfrac{A_{12}}{\sqrt{Z_{01}Z_{02}}}\\
                \sqrt{Z_{01}Z_{02}}A_{21} & \dfrac{\sqrt{Z_{01}}A_{22}}{\sqrt{Z_{02}}}
            \end{bmatrix}
        \end{equation}
    \end{theorem}

    \begin{tcbproof}\begin{proof}
        根据A参数矩阵的定义和阻抗归一化波的概念就可以推导出阻抗归一化A参数矩阵的表达式。将归一化电压和电流表示为:
        \begin{equation}
            \begin{aligned}
                \begin{bmatrix}
                    \bar{u}_1\\
                    \bar{i}_1
                \end{bmatrix}
                &=\begin{bmatrix}
                    \dfrac{u_1}{\sqrt{Z_{01}}}\\
                    i_1\sqrt{Z_{01}}
                \end{bmatrix}\\
                &=\begin{bmatrix}
                    \dfrac{A_{11}}{\sqrt{Z_{01}}}u_2+\dfrac{A_{12}}{Z_{01}}i_2\\
                    A_{21}\sqrt{Z_{01}}u_2+A_{22}\sqrt{Z_{01}}i_2
                \end{bmatrix}\\
                &=\begin{bmatrix}
                    \dfrac{\sqrt{Z_{02}}A_{11}}{\sqrt{Z_{01}}}\bar{u}_2+\dfrac{A_{12}}{\sqrt{Z_{01}Z_{02}}}\bar{i}_2\\
                    A_{21}\sqrt{Z_{01}Z_{02}}\bar{u}_2+\dfrac{\sqrt{Z_{01}}A_{22}}{\sqrt{Z_{02}}}\bar{i}_2
                \end{bmatrix}\\
                &=\begin{bmatrix}
                    \dfrac{\sqrt{Z_{02}}A_{11}}{\sqrt{Z_{01}}} & \dfrac{A_{12}}{\sqrt{Z_{01}Z_{02}}}\\
                    A_{21}\sqrt{Z_{01}Z_{02}} & \dfrac{\sqrt{Z_{01}}A_{22}}{\sqrt{Z_{02}}}
                \end{bmatrix}
                \begin{bmatrix}
                    \bar{u}_2\\
                    \bar{i}_2
                \end{bmatrix}
                =\begin{bmatrix}
                    a_{11}&a_{12}\\
                    a_{21}&a_{22}
                \end{bmatrix}
                \begin{bmatrix}
                    \bar{u}_2\\
                    \bar{i}_2                
                \end{bmatrix}
                =\bm{a}
                \begin{bmatrix}
                    \bar{u}_2\\
                    \bar{i}_2                
                \end{bmatrix}
            \end{aligned}
        \end{equation}
        其中$\bm{a}$表示阻抗归一化的A参数矩阵。        
    \end{proof}\end{tcbproof}

    \begin{theorem}
    {A参数矩阵转化为S参数矩阵}
    {A参数矩阵转化为S参数矩阵}
        归一化后的A参数矩阵转S参数矩阵:
        \begin{equation}
            \bm{S}=\frac{1}{a_{11}+a_{12}+a_{21}+a_{22}}
            \begin{bmatrix}
                a_{11}+a_{12}-a_{21}-a_{22}&2|\bm{a}|\\
                2&a_{12}+a_{22}-a_{21}-a_{11}
            \end{bmatrix}
        \end{equation}
    \end{theorem}
    \begin{tcbproof}\begin{proof}
        A参数矩阵同二端口网络的S参数矩阵的关系,本质上就是电压电流同归一化入射波和散射波的关系:
        \begin{subequations}\label{Equ: 电压电流同归一化入射波和散射波的关系}
            \begin{numcases}{\mbox{}} 
                \begin{bmatrix}
                    \bar{u}_1\\
                    \bar{i}_1
                \end{bmatrix}
                =\begin{bmatrix}
                    a_1+b_1\\
                    a_1-b_1
                \end{bmatrix}
                =\begin{bmatrix}
                    1&1\\
                    1&-1
                \end{bmatrix}
                \begin{bmatrix}
                    a_1\\
                    b_1
                \end{bmatrix}\\
                \begin{bmatrix}
                    \bar{u}_2\\
                    \bar{i}_2
                \end{bmatrix}
                =\begin{bmatrix}
                    a_2+b_2\\
                    -(a_2-b_2)
                \end{bmatrix}
                =\begin{bmatrix}
                    1&1\\
                    -1&1
                \end{bmatrix}
                \begin{bmatrix}
                    a_2\\
                    b_2
                \end{bmatrix}
            \end{numcases}
        \end{subequations}
        注意电流的方向为\hyperref[Fig: A参数矩阵的阻抗归一化]{图\ref*{Fig: A参数矩阵的阻抗归一化}}中标注的方向。

        把\hyperref[Equ: 电压电流同归一化入射波和散射波的关系]{式(\ref*{Equ: 电压电流同归一化入射波和散射波的关系})}带入归一化A参数矩阵的定义式中,再整理成$\bm{b}=\bm{S}\bm{a}$的形式,就可以得证。

        反过来,把$a,b$用电压和电流表示,带入到$\bm{b}=\bm{S}\bm{a}$中,再整理成A参数矩阵定义式的形式,就可以得到S参数矩阵转A参数矩阵的方式。
    \end{proof}\end{tcbproof}

    \subsection{输入阻抗与S参数的关系}
    根据$a,b$和电压电流的关系,不仅可以导出A参数矩阵和S参数矩阵的关系,还可以推导出输入阻抗和S参数的关系。

    仍以\hyperref[Fig: A参数矩阵的阻抗归一化]{图\ref*{Fig: A参数矩阵的阻抗归一化}}所示的\circled{1}端口为例,
    \begin{equation}
        Z_{in}=\frac{u_1}{i_1}
        =\frac{(a_1+b_1)\sqrt{Z_{01}}}{(a_1-b_1)/\sqrt{Z_{01}}}
        =Z_{01}\frac{a_1+b_1}{a_1-b_1}
    \end{equation}

    对于一般性的微波网络而言,$b_1$可能由各个端口入射波贡献,上式无法继续化简。

    但是对于端口间高度隔离的网络,有$S_{12}=S_{13}=\cdots=S_{1n}=0$,此时有$b_1=S_{11}a_1$。那么,
    \begin{equation}
        Z_{in}=Z_{01}\frac{1+S_{11}}{1-S_{11}}
    \end{equation}

    换言之,对于任何一个微波网络,如果从它的其他所有端口入射的能量都不能耦合或者输出到\circled{$p$}端口,(意味着该网络的S参数矩阵的第$p$行除了$S_{pp}外$外都是0或极小)那么,\circled{$p$}端口的输入阻抗只与$S_{pp}$以及特性阻抗有关。否则,该端口的阻抗匹配将会十分复杂。


\section{二端口元件}
二端口元件(双端口元件)是应用最多的一种元件。
\begin{figure}[h!]
    \begin{center}
        \begin{tikzpicture}[
                circuit ee IEC,
                % x=3cm,y=2cm,
                semithick,
                every info/.style={font=\footnotesize},
                small circuit symbols,
                % set resistor graphic=var resistor IEC graphic,
                % set diode graphic=var diode IEC graphic,
                set make contact graphic = var make contact IEC graphic,
                circuit declare symbol= module, set module graphic=
                    {draw, shape=rectangle, minimum width=4cm, minimum height=2cm, text width = 2cm, align = flush center, fill=white},
                ]

            \node [module] (网络A) at (0,0) {二端口网络};

            \node [circle, draw=black, inner sep=1pt, minimum size=3pt] (P1far_N) at (-4.0,-0.8) {};
            \node [circle, draw=black, inner sep=1pt, minimum size=3pt] (P1far_P) at (-4.0,0.8) {};
            \node [circle, draw=black, inner sep=1pt, minimum size=3pt] (P2far_N) at (4.0,-0.8) {};
            \node [circle, draw=black, inner sep=1pt, minimum size=3pt] (P2far_P) at (4.0,0.8) {};
            
            \node [] (P1near_N) at (-1.9,-0.8) {};
            \node [] (P1near_P) at (-1.9,0.8) {};
            \node [] (P2near_N) at (1.9,-0.8) {};
            \node [] (P2near_P) at (1.9,0.8) {};

            \draw (P1far_P) to (P1near_P);
            \draw (P1far_N) to (P1near_N);
            \draw (P2far_P) to (P2near_P);
            \draw (P2far_N) to (P2near_N);

            \draw[->, color=red!67] (P1far_P) ++(0.5cm,0.25cm) -- node[above]{$a_1$} ++(0.9cm,0);
            \draw[->, color=blue!67] (P2near_P) ++(0.5cm,0.25cm) -- node[above]{$b_2$} ++(0.9cm,0);
            \draw[->, color=blue!67] (P1near_N) ++(-0.5cm,-0.25cm) -- node[below]{$b_1$} ++(-0.9cm,0);
            \draw[->, color=red!67] (P2far_N) ++(-0.5cm,-0.25cm) -- node[below]{$a_2$} ++(-0.9cm,0);
            
            \draw (-3,0) node[] {\circled{1}};
            \draw (+3,0) node[] {\circled{2}};

            \node at (8,0) {$\displaystyle \begin{bmatrix}
                    b_1\\b_2
                \end{bmatrix}=\begin{bmatrix}
                    S_{11}&S_{12}\\
                    S_{21}&S_{22}
                \end{bmatrix}\begin{bmatrix}
                    a_1\\a_2
                \end{bmatrix}$};
        \end{tikzpicture}
    \end{center}
    \caption{\kaishu 二端口元件}\label{Fig: 二端口元件}
\end{figure}

在讨论二端口元件的性质时,我们假设\circled{1}是输入端口,靠近源端;\circled{2}是输出端口,靠近负载。

    \subsection{二端口网络的负载变换}
    \begin{theorem}
    {二端口网络的反射系数变换}
    {二端口网络的反射系数变换}
        将二端口网络的端口\circled{2}接一负载网络,负载的输入阻抗为$Z_L$,反射系数为$\varGamma_L$,则在经过一个由S参数描述的二端口网络之后,产生如下变换:
        \begin{equation}
            \varGamma_\mathrm{in}=S_{11}+\frac{S_{12}S_{21}\varGamma_L}{1-S_{22}\varGamma_L}
        \end{equation}
        其中,$\varGamma_{in}=\frac{b_1}{a_1}$为从端口\circled{1}看向负载的反射系数,$\varGamma_L=\frac{a_2}{b_2}$为负载反射系数。        
    \end{theorem}



    \begin{tcbproof}
    % \begin{proof}
        关键问题在于理解
        \begin{subequations}
            \begin{numcases}{} 
                \varGamma_{in}=\frac{b_1}{a_1}\\
                \varGamma_L=\frac{a_2}{b_2}
            \end{numcases}
        \end{subequations}
        剩下的推导过程只需要将$\bm{b}=\bm{S}\bm{a}$带入上面两个方程,解出$a_1$即可。

        于是,为什么反射系数可以用归一化入射波$a$和散射波$b$表示?

        \begin{wrapfigure}[7]{r}{0.5\textwidth}
            \vspace{-20pt}
            \begin{center}
                \begin{tikzpicture}[
                        scale=0.7,
                        circuit ee IEC,
                        % x=3cm,y=2cm,
                        semithick,
                        every info/.style={font=\footnotesize},
                        small circuit symbols,
                        % set resistor graphic=var resistor IEC graphic,
                        % set diode graphic=var diode IEC graphic,
                        circuit declare symbol= module, set module graphic=
                            {draw, shape=rectangle, minimum width=1.4cm, minimum height=1.4cm, text width = 1.4cm, align = flush center, fill=white},
                        ]

                    \node [module] (网络A) at (0,0) {$[\bm{S}]$};

                    \node [circle, draw=black, inner sep=1pt, minimum size=3pt, label=left:{\color{gray!50!black}$-$}] (P1far_N) at (-2.9,-0.8) {};
                    \node [circle, draw=black, inner sep=1pt, minimum size=3pt, label=left:{\color{gray!50!black}$+$}] (P1far_P) at (-2.9,0.8) {};

                    \node [inner sep=0, outer sep=0] (P2far_N) at (3.4,-0.8) {};
                    % \node [] (P2far_P) at (2.9,0.8) {};
                    
                    \node [] (P1near_N) at (-0.85,-0.8) {};
                    \node [] (P1near_P) at (-0.85,0.8) {};
                    \node [] (P2near_N) at (0.85,-0.8) {};
                    \node [] (P2near_P) at (0.85,0.8) {};


                    \draw (P1far_P) to (P1near_P);
                    \draw (P1far_N) to (P1near_N);
                    % \draw (P2far_P) to (P2near_P);
                    % \draw (P2far_N) to (P2near_N);


                    % \draw (P2near_N) to  (P2far_N) to [resistor={fill=white}] ++(up:1.6) to  (P2near_P) ;
                    \draw (P2near_N) to [resistor={circuit symbol size=width 5 height 0.6}] ++(right:3.3) to [resistor={circuit symbol size=width 3 height 2, fill=white, info=center:L}] ++(up:1.6) to [resistor={circuit symbol size=width 5 height 0.6}] (P2near_P) ;


                    \draw[->, color=red!67] (P1far_P) ++(0.5cm,-0.25cm) -- node[below]{$a_1$} ++(0.9cm,0);
                    \draw[->, color=blue!67] (P2near_P) ++(0.5cm,+0.25cm) -- node[above]{$b_2$} ++(0.9cm,0);
                    \draw[->, color=blue!67] (P1near_N) ++(-0.5cm,-0.25cm) -- node[below]{$b_1$} ++(-0.9cm,0);
                    \draw[->, color=red!67] (P2near_N) ++(1.4cm,+0.25cm) -- node[above]{$a_2$} ++(-0.9cm,0);

                    \draw[->,>=stealth] (P1far_P) ++(0.5cm,0) -- node[above, near end]{\color{gray!50!black} $i_1$} ++(0.5cm,0);
                    \draw (-3.8,0) node[] {\color{gray!50!black} $u_1$};
                    
                    \path (P1far_N) ++ (-0.5em,-1.2em) node[anchor=north, color=gray!50!black] (GammaIN){$\varGamma_{in}$};
                    \draw[-{[sep=3pt]>}, >=stealth, inner sep=3pt, color=gray!50!black] ([color=gray!50!black]GammaIN.north) |- ++(0.8,1.2);

                    \path (P2far_N) ++ (0,-1.2em) node[anchor=north, color=gray!50!black] (GammaIN){$\varGamma_L$};
                    \draw[-{[sep=3pt]>}, >=stealth, inner sep=3pt, color=gray!50!black] ([color=gray!50!black]GammaIN.north) |- ++(0.4,1.2);

                    \end{tikzpicture}                 
            \end{center}
            \vspace{-20pt}
            \caption{用散射波和入射波表示反射系数}\label{Fig: 用散射波和入射波表示反射系数}
        \end{wrapfigure} 

        根据反射系数定义,
        \begin{equation}
            \varGamma_{in}=\frac{u_1^-}{u_1^+}
        \end{equation}
        或者根据\hyperref[Equ: Reflection Coefficient of current at z']{式(\ref*{Equ: Reflection Coefficient of current at z'})}:
        \begin{equation}
            \varGamma_{in}=\frac{-i_1^-}{i_1^+}
        \end{equation}
        注意,$u_1$和$i_1$表示网络端口处的电压和电流,参考面没有选在传输线上。加$+,-$上标也是表示在端口处的入射分量和反射分量。那么,根据归一化入射波和散射波的定义,$a_1=\frac{u_1^+}{\sqrt{Z_0}}={i_1^+}{\sqrt{Z_0}},b_1=\frac{u_1^-}{\sqrt{Z_0}}=-{i_1^-}{\sqrt{Z_0}}$。
        所以
        \begin{equation}
            \varGamma_{in}
            =\frac{u_1^-/\sqrt{Z_0}}{u_1^+/\sqrt{Z_0}}
            =\frac{\frac{u_1^-}{\sqrt{Z_0}}-i_1^-\sqrt{Z_0}}{\frac{u_1^+}{\sqrt{Z_0}}+i_1^+\sqrt{Z_0}}
            =\frac{b_1}{a_1}
        \end{equation}

        类似地,如果二端口网络的输出端口直接和负载相连,可以得到
        \begin{equation}
            \varGamma_L
            =\frac{u_2^-/\sqrt{Z_0}}{u_2^+/\sqrt{Z_0}}
            =\frac{a_2}{b_2}
        \end{equation}

        但是,如果输出端口到负载之间有一段长$\Delta z$的传输线,如\hyperref[Fig: 用散射波和入射波表示反射系数]{图\ref*{Fig: 用散射波和入射波表示反射系数}}所示,则上式就不再成立。

        因为$\varGamma_L$定义的是负载处的反射电压比上负载处的前向电压,而$a_2$和$b_2$则是端口处的入射波和散射波。
        此时关系变为:
        \begin{equation}
            \varGamma_L
            =\frac{u_2^-\mathrm{e}^{\gamma \Delta z}}{u_2^+\mathrm{e}^{-\gamma \Delta z}}
            =\frac{a_2}{b_2}\mathrm{e}^{2\gamma \Delta z}
        \end{equation}
        其中,$\gamma$为传输线的传播常数。
    % \end{proof}
    \end{tcbproof}

    \begin{corollary}
    {无耗二端口网络的匹配定理}
    {无耗二端口网络的匹配定理}
        一个无耗的二端口网络,输出端口接一反射系数$\varGamma_L$的负载。则当且仅当
        \begin{equation}
            S_{22}=\varGamma_L^*
        \end{equation}
        时,从该网络的输入端口看向负载,阻抗匹配。
    \end{corollary}

    \begin{tcbproof}
        \begin{wrapfigure}[4]{r}{0.5\textwidth}
            \vspace{-20pt}
            \begin{center}
                \begin{tikzpicture}[
                        scale=0.7,
                        circuit ee IEC,
                        % x=3cm,y=2cm,
                        semithick,
                        every info/.style={font=\footnotesize},
                        small circuit symbols,
                        % set resistor graphic=var resistor IEC graphic,
                        % set diode graphic=var diode IEC graphic,
                        circuit declare symbol= module, set module graphic=
                            {draw, shape=rectangle, minimum width=1.4cm, minimum height=1.4cm, text width = 1.4cm, align = flush center, fill=white},
                        ]

                    \node [module] (网络A) at (0,0) {$[\bm{S}]$};

                    \node [circle, draw=black, inner sep=1pt, minimum size=3pt] (P1far_N) at (-2.9,-0.8) {};
                    \node [circle, draw=black, inner sep=1pt, minimum size=3pt] (P1far_P) at (-2.9,0.8) {};

                    \node [inner sep=0, outer sep=0] (P2far_N) at (3.4,-0.8) {};
                    % \node [] (P2far_P) at (2.9,0.8) {};
                    
                    \node [] (P1near_N) at (-0.85,-0.8) {};
                    \node [] (P1near_P) at (-0.85,0.8) {};
                    \node [] (P2near_N) at (0.85,-0.8) {};
                    \node [] (P2near_P) at (0.85,0.8) {};


                    \draw (P1far_P) to (P1near_P);
                    \draw (P1far_N) to (P1near_N);
                    % \draw (P2far_P) to (P2near_P);
                    % \draw (P2far_N) to (P2near_N);

                    % \draw (P2near_N) to  (P2far_N) to [resistor={fill=white}] ++(up:1.6) to  (P2near_P) ;
                    \draw (P2near_N) to  ++(right:2.5) to [resistor={circuit symbol size=width 3 height 2, fill=white, info=center:L}] ++(up:1.6) to  (P2near_P) ;

                    
                    \path (P1far_N) ++ (-0.5em,-1.2em) node[anchor=north, color=gray!50!black] (GammaIN){$\varGamma_{in}=0$};
                    \draw[-{[sep=3pt]>}, >=stealth, inner sep=3pt, color=gray!50!black] ([color=gray!50!black]GammaIN.north) |- ++(0.8,1.2);

                    \path (P2far_N) ++ (-2.8em,-1.2em) node[anchor=north, color=gray!50!black] (GammaIN){$\varGamma_L\neq0$};
                    \draw[-{[sep=3pt]>}, >=stealth, inner sep=3pt, color=gray!50!black] ([color=gray!50!black]GammaIN.north) |- ++(0.8,1.2);

                    \end{tikzpicture}                 
            \end{center}
            \vspace{-20pt}
            \caption{无耗二端口网络用于阻抗匹配}\label{Fig: 无耗二端口网络用于阻抗匹配}
        \end{wrapfigure} 
        要使输入端口阻抗匹配,则
        \begin{equation}
            \varGamma_\mathrm{in}=S_{11}+\frac{S_{12}S_{21}\varGamma_L}{1-S_{22}\varGamma_L}=0
        \end{equation}

        用模值和辐角表示:
        \begin{equation}
            |S_{11}|\mathrm{e}^{\mathrm{j}\varphi_{11}}-|S_{11}||S_{22}|\mathrm{e}^{\mathrm{j}(\varphi_{11}+\varphi_{22})}\varGamma_L+|S_{12}||S_{21}|\mathrm{e}^{\mathrm{j}(\varphi_{12}+\varphi_{21})}\varGamma_L=0
        \end{equation}

        根据二端口网络的无耗性质 \hyperref[Equ: 无耗二端口网络的S参数]{式(\ref*{Equ: 无耗二端口网络的S参数})},得到
        \begin{equation}
            |S_{11}|\mathrm{e}^{\mathrm{j}\varphi_{11}}+\varGamma_L\left(-|S_{11}|^2\mathrm{e}^{\mathrm{j}(\varphi_{12}+\varphi_{21})+\mathrm{j}\pi}+|S_{12}|^2\mathrm{e}^{\mathrm{j}(\varphi_{12}+\varphi_{21})}\right)=0
        \end{equation}
        带入无耗网络的振幅条件$|S_{11}|^2+|S_{12}|^2=1$,合并得
        \begin{equation}
            |S_{11}|\mathrm{e}^{\mathrm{j}\varphi_{11}}+|\varGamma_L|\mathrm{e}^{\mathrm{j}(\varphi_{12}+\varphi_{21})+\mathrm{j}\varphi_l}=0
        \end{equation}
        再用一次相位条件化简
        \begin{equation}
            |S_{11}|\mathrm{e}^{\mathrm{j}\varphi_{11}}=|\varGamma_L|\mathrm{e}^{\mathrm{j}(\varphi_{11}+\varphi_{22})+\mathrm{j}\varphi_l}
        \end{equation}
        即
        \begin{equation}
            |S_{11}|\mathrm{e}^{-\mathrm{j}\varphi_{22}}
            =|S_{22}|\mathrm{e}^{-\mathrm{j}\varphi_{22}}
            =S_{22}^*
            =|\varGamma_L|\mathrm{e}^{\mathrm{j}\varphi_l}
            =\varGamma_L
        \end{equation}
    \end{tcbproof}

    % \begin{corollary}
        
    % \end{corollary}
\section{单端口元件}
    单端口元件常作为负载使用。它的S参数矩阵只有$S_{11}$一个元素。
    \begin{equation}
        b_1=S_{11}a_1
    \end{equation}
    此时,
    \begin{equation}
        \varGamma_{in}=\frac{b_1}{a_1}=S_{11}
    \end{equation}
    