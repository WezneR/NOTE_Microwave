% !TeX root = Microwave.tex
\chapter{引用}
\section{省略的证明过程}
\begin{enumerate}
    \renewcommand*\labelenumi{\circled{\theenumi}}
    \item 函数是从数到数的映射。
    \item 泛函是从函数到数的映射。
    \item 算子是从函数到函数的映射。
\end{enumerate}
\section{坐标系}
    \subsection{广义正交坐标系}
    
    与直角坐标系不同,正交曲线坐标系的单位基向量$\hat{e}_i$通常不是常矢量。它们模始终为1,但方向可能随空间中一点位置不同而变化。

    \begin{definition}{正交曲线坐标系}{正交曲线坐标系}
        三个有序数$(u_1,u_2,u_3)$,满足
        \begin{enumerate}
            \renewcommand*\labelenumi{(\theenumi)}
            \item $u_i=u_i(x,y,z)$是直角坐标$(x,y,z)$的单值函数;
            \item $\hat{e}_1,\hat{e}_2,\hat{e}_3$是曲线$u_1,u_2,u_3$的切线上指向$u_1,u_2,u_3$增大一方的单位矢量,它们两两正交,且满足右手螺旋关系:
            \begin{equation*}
                \hat{e}_1\times \hat{e}_2=\hat{e}_3\,,\;
                \hat{e}_2\times \hat{e}_3=\hat{e}_1\,,\;
                \hat{e}_3\times \hat{e}_1=\hat{e}_2
            \end{equation*}
        \end{enumerate}
        则称$(u_1,u_2,u_3)$构成的坐标系为正交曲线坐标系。
    \end{definition}



    正交坐标系使用曲线坐标$(u_1,u_2,u_3)$,局部单位基向量为$\hat{e}_1,\hat{e}_2,\hat{e}_3$,$h_i,\,i=1,2,3$为对应的拉梅(\emph{Lame})系数。$\varPhi$为标量,矢量$\vec{A}=(A_1,A_2,A_3)$则:
    \begin{subequations}
        \begin{align}
            \nabla \varPhi&=\frac{1}{h_1}\frac{\partial \varPhi}{\partial u_1}\hat{e}_1+\frac{1}{h_2}\frac{\partial \varPhi}{\partial u_2}\hat{e}_2+\frac{1}{h_3}\frac{\partial \varPhi}{\partial u_3}\hat{e}_3\\
            \nabla\cdot\vec{A}&=\frac{1}{h_1h_2h_3}\left[\frac{\partial }{\partial u_1}(h_2h_3A_1)+\frac{\partial }{\partial u_2}(h_1h_3A_2)+\frac{\partial }{\partial u_3}(h_1h_2A_3)\right]\\
            \nabla\times\vec{A}&=\frac{1}{h_1h_2h_3}\begin{vmatrix}
            h_1\hat{e}_1&h_2\hat{e}_2&h_3 \hat{e}_3\\
            \frac{\partial }{\partial u_1}&\frac{\partial }{\partial u_2}&\frac{\partial }{\partial u_3}\\
            h_1A_1&h_2A_2&h_3 A_3
        \end{vmatrix}
        \end{align}
    \end{subequations}

    \subsection{\emph{Lame}系数}
    \paragraph{如何求拉梅系数?}
    “拉梅系数是坐标曲线的一组特殊切向量的模。”
    \begin{equation}
        h_i=|\bm{h}_i|
    \end{equation}
    其中
    \begin{equation}
        \bm{h}_i=\frac{\partial \bm{x}}{\partial u_i}=\left(\frac{\partial x_1}{\partial u_i},\frac{\partial x_2}{\partial u_i},\frac{\partial x_3}{\partial u_i}\right),\;i=1,2,3
    \end{equation}
    只有当切向量$\bm{h}_1,\bm{h}_2,\bm{h}_3$两两正交时,曲线坐标系是曲线正交坐标系,才有拉梅系数定义。
    其中$\bm{x}=(x_1,x_2.x_3)$为空间中一点的笛卡尔坐标系描述,$(u_1,u_2,u_3)$为该曲线坐标系对该点的描述。\\
    这里只考虑了3维的情况:涉及拉梅系数,就一定涉及到一个特定3维空间的广义正交坐标系表示和笛卡尔坐标系表示之间的关系。

    例如:
    \begin{equation}
        \mbox{柱坐标系}\left\{\begin{aligned}
            &\bm{h}_1=\left(\frac{\partial \cos\phi}{\partial \rho},\frac{\partial \sin\phi}{\partial \rho},\frac{\partial z}{\partial \rho}\right)\\
            &\bm{h}_2=\left(\frac{\partial \cos\phi}{\partial \phi},\frac{\partial \sin\phi}{\partial \phi},\frac{\partial z}{\partial \phi}\right)\\
            &\bm{h}_3=\left(\frac{\partial \cos\phi}{\partial z},\frac{\partial \sin\phi}{\partial z},\frac{\partial z}{\partial z}\right)
        \end{aligned}\right.
        \Rightarrow
        \left\{\begin{aligned}
            &h_1=|\left(\cos\phi,\sin\phi,0\right)|=1\\
            &h_2=|\left(-\rho\sin\phi,\rho\cos\phi,0\right)|=\rho\\
            &h_3=|\left(0,0,1\right)|=1
        \end{aligned}\right.
    \end{equation}
    \begin{equation}
        \begin{aligned}
            \mbox{球坐标系}\left\{\begin{aligned}
                &\bm{h}_1=\left(\frac{\partial r\sin\theta\cos\phi}{\partial r},\frac{\partial r\sin\theta\sin\phi}{\partial r},\frac{\partial r\cos\theta}{\partial r}\right)\\
                &\bm{h}_2=\left(\frac{\partial r\sin\theta\cos\phi}{\partial \theta},\frac{\partial r\sin\theta\sin\phi}{\partial \theta},\frac{\partial r\cos\theta}{\partial \theta}\right)\\
                &\bm{h}_3=\left(\frac{\partial r\sin\theta\cos\phi}{\partial \phi},\frac{\partial r\sin\theta\sin\phi}{\partial \phi},\frac{\partial r\cos\theta}{\partial \phi}\right)
            \end{aligned}\right.\\
            \Rightarrow
            \left\{\begin{aligned}
                &h_1=|\left(\sin\theta\cos\phi,\sin\theta\sin\phi,\cos\theta\right)|=1\\
                &h_2=|\left(r\cos\theta\cos\phi,r\cos\theta\cos\phi,-r\sin\theta\right)|=r\\
                &h_3=|\left(-r\sin\theta\sin\phi,r\sin\theta\cos\phi,0\right)|=r\sin\theta
            \end{aligned}\right.
        \end{aligned}
    \end{equation}
\section{公式与定理}
    \subsection{华里士公式}
    \begin{equation}
        \int_{0}^{\frac{\pi}{2}}\sin^nx\,\mathrm{d}x=
        \left\{
            \begin{aligned}
                &\frac{n-1}{n}\cdot\frac{n-3}{n-2}\cdot\,\cdots\,\cdot\frac{3}{4}\cdot\frac{1}{2}{\,\color{red}\cdot\,\frac{\pi}{2}}\;,\quad \mathrm{even} ~n\\
                &\frac{n-1}{n}\cdot\frac{n-3}{n-2}\cdot\,\cdots\,\cdot\frac{4}{5}\cdot\frac{2}{3}\;,\quad \mathrm{odd} ~n
            \end{aligned}
        \right.
    \end{equation}

    \subsection{矢量恒等式}
    \begin{subequations}
        \begin{align}
            &\nabla\cdot(\nabla\times\vec{A})=0&\mbox{“旋无源”}\\
            &\nabla\times(\nabla \varphi)=\vec{0}&\mbox{“梯无旋”}\\
            &\nabla\times \nabla\times\vec{A}=\nabla(\nabla\cdot\vec{A})-\nabla^2 \vec{A}&
        \end{align}
    \end{subequations}

    \subsection{斯托克斯公式}
    设矢量$\vec{A}\in \mathbb{R}^3$的三个分量都在有向曲面$S$及其边界$\partial S=\Gamma$上具有连续的一阶偏导数,且$\Gamma$的正向与$S$的侧符合右手规则,则:
    \begin{equation}
        \iint_S \nabla\times\vec{A}\cdot\mathrm{d}\vec{S}
        =\oint_\Gamma \vec{A}\cdot\mathrm{d}\vec{r}
    \end{equation}
